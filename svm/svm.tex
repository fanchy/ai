%!TEX program = xelatex
% This is a small sample LaTeX input file (Version of 10 April 1994)
%
% Use this file as a model for making your own LaTeX input file.
% Everything to the right of a  %  is a remark to you and is ignored by LaTeX.
 
% The Local Guide tells how to run LaTeX.
 
% WARNING!  Do not type any of the following 10 characters except as directed:
%                &   $   #   %   _   {   }   ^   ~   \   
 
%\documentclass{article}        % Your input file must contain these two lines 
\documentclass[12pt, a4paper, oneside]{ctexart}
\usepackage{xeCJK}

\usepackage{indentfirst, abstract, appendix} %首行缩进
\usepackage{graphicx} %插入图片
\usepackage{amsmath, amssymb, geometry} 
\usepackage{listings, xcolor} %代码高亮
\graphicspath{{../graphics/}}
\linespread{1.2}
\geometry{left=2.5cm, right=2.5cm, top=2.5cm, bottom=2.5cm}
\title{\textbf{SVM的数学推导和Python实现}}
\author{赵新锋}
\date{\today}
\renewcommand{\abstractname}{\Large\textbf{摘要}}
\pagestyle{plain}

\lstset{ %
language=Python,                % the language of the code
basicstyle=\footnotesize,           % the size of the fonts that are used for the code
%numbers=left,                   % where to put the line-numbers
%numberstyle=\tiny\color{gray},  % the style that is used for the line-numbers
stepnumber=2,                   % the step between two line-numbers. If it's 1, each line 
                                % will be numbered
numbersep=5pt,                  % how far the line-numbers are from the code
backgroundcolor=\color{white},      % choose the background color. You must add \usepackage{color}
showspaces=false,               % show spaces adding particular underscores
showstringspaces=false,         % underline spaces within strings
showtabs=false,                 % show tabs within strings adding particular underscores
frame=single,                   % adds a frame around the code
rulecolor=\color{black},        % if not set, the frame-color may be changed on line-breaks within not-black text (e.g. commens (green here))
tabsize=2,                      % sets default tabsize to 2 spaces
captionpos=b,                   % sets the caption-position to bottom
breaklines=true,                % sets automatic line breaking
breakatwhitespace=false,        % sets if automatic breaks should only happen at whitespace
title=\lstname,                 % show the filename of files included with \lstinputlisting;
                                % also try caption instead of title
keywordstyle=\color{blue},          % keyword style
commentstyle=\it\color[RGB]{0,96,96},                % 设置代码注释的格式
stringstyle=\rmfamily\slshape\color[RGB]{128,0,0},         % string literal style
escapeinside={\%*}{*)},            % if you want to add LaTeX within your code
morekeywords={*,...}               % if you want to add more keywords to the set
}
\begin{document}               % plus the \end{document} command at the end.
\maketitle

\setcounter{page}{0}
\maketitle
\thispagestyle{empty}

\begin{abstract}
支持向量机(support vector machines, SVM)是一种分类模型,该模型在特征空间中求解间隔最大的分类超平面。当训练数据近似线性
可分时,可以通过增加软间隔学习一个线性分类器。当线性不可分时,利用核技巧,隐式的将特征空间映射到高维特征空间,从而达到线性
可分。使用序列最小最优化算法(SMO),可以快速求解模型的参数。
\par\textbf{关键词:}支持向量机; SVM; SMO; 矩阵运算; 矩阵求导;numpy;sklearn. 
\end{abstract}

\newpage
\pagenumbering{Roman}
\setcounter{page}{1}
\tableofcontents
\newpage
\setcounter{page}{1}
\pagenumbering{arabic}


\newpage
\section{数学推导与python实现}

\subsection{分类超平面}
当训练数据线性可分,可以得到一个线性超平面 $ x \cdot w + b = 0 $,将在超平面上方的归为正类,将在超平面下方的归为负类。当数据点
与超平面距离越远时,表示分类的确定性越高,这样虽然线性分类的超平面可能有无数多个,但是我们可以找到一个所有点距离超平面最大的一个
超平面。相应的决策函数为:
\begin{align*}
    f(x) = sign(x \cdot w^* + b^*)
\end{align*}
\begin{figure}[htbp]
    \centering
    \includegraphics[width=14cm]{svm1.jpg}
    \caption{SVM线性可分}\label{fig1}
\end{figure}



\subsection{公式推导}
$x_i^T \cdot w_0 + b_0$ 为正时,$y_i$为正,当$x_i^T \cdot w_0 + b_0$ 为负时,$y_i$为负,则可以定义
$\frac{y_i \cdot (x_i^T \cdot w_0 + b_0)}{ \left\|w_0\right\|}$为几何间隔,表示数据点距离超平面的距离。
模型最终会找到一个参数为$w_0$和$b_0$的分离超平面,所有点距离超平面的距离都大于等于d,将距离正好等于d的数据
点称之为支持向量。
\begin{align}
    \mathop{\arg\max_{w_0, b_0}} \ d &= \frac{y_0 \cdot (x_0^T \cdot w_0 + b_0)}{ \left\|w_0\right\|} 			\label{eq1}\\
	\mathrm{ s.t. }\ \   &\frac{y_i \cdot (x_i^T \cdot w_0 + b_0)}{ \left\|w_0\right\|} \geq d				\nonumber
\end{align}
将 $w_0$ 和 $b_0$ 进行一定比例的缩放
\begin{align}
	w &= \frac{w_0 }{ y_0 \cdot (x_0^T \cdot w_0 + b_0)} 			\nonumber\\
    b &= \frac{b_0 }{ y_0 \cdot (x_0^T \cdot w_0 + b_0)} 			\nonumber
\end{align}
可以将 \eqref{eq1} 式化简为:
\begin{align}
	\mathop{\arg\max_{w}} \ d &= \frac{1}{ \left\|w\right\|} 			\nonumber\\
    \iff \mathop{\arg\min_{w}}  \ d &=  \left\|w\right\| \nonumber\\
    \iff \mathop{\arg\min_{w}}  \ d &=  \frac{1}{2}w^T \cdot w \label{eq2}\\
    \mathrm{ s.t. }\ \   & y_i \cdot (x_i^T \cdot w + b) \geq 1				\label{eq3}
\end{align}
将\eqref{eq2} 和 \eqref{eq3} 利用拉格朗日乘数法,获得拉格朗日原始问题形式:
\begin{align}
    \mathop{\arg\min_{w,b}\max_{\alpha}} \ 	L(w, b, {\alpha}) &= \frac{1}{2}	w^T \cdot w  - {\alpha} ^T \cdot  (y \odot (X \cdot w + b) - 1) \nonumber\\
        &= \frac{1}{2}	w^T \cdot w  - (\alpha \odot y) ^T \cdot  (X \cdot w + b) + \alpha^T \cdot \boldsymbol{1}^m \nonumber\\
        &= \frac{1}{2}	w^T \cdot w  - (\alpha \odot y) ^T \cdot  (X \cdot w + b) + \boldsymbol{1}^T\cdot\alpha \label{eq4}
\end{align}
当满足KTT条件时, 拉格朗日对偶问题的解等价于\eqref{eq4} 的解:
\begin{align}
    \mathop{\arg\max_{\alpha}\min_{w,b}} \ 	L(w, b, {\alpha}) &= \frac{1}{2}	w^T \cdot w  - (\alpha \odot y) ^T \cdot  (X \cdot w + b) + \boldsymbol{1}^T\cdot\alpha \label{eq5}
\end{align}
首先求L以 w、b为参数的极小值,通过求微分得到偏导数形式。
\begin{align}
    \mathrm{d}L &= \frac{1}{2}tr[(\mathrm{d}w)^T \cdot w + w^T \cdot \mathrm{d}w) - (\alpha \odot y) ^T \cdot(X\mathrm{d}w) \nonumber \\
                & \ \ \  - (1^T\cdot(\alpha \odot y)) ^T \cdot \mathrm{d}b \nonumber\\
                & \ \ \  - (\mathrm{d}\alpha)^T \cdot (y \odot (X \cdot w + b) - 1)] \nonumber \\
                &= tr[w^T\mathrm{d}w - (X^T\cdot(\alpha \odot y))^T\mathrm{d}w - (\alpha \odot y) ^T \cdot \mathrm{d}b - (y \odot (X \cdot w + b) - 1)^T\mathrm{d}\alpha] \nonumber
\end{align}
从而得到w、b 偏导数,并令偏导数为0。
\begin{align}
    \frac{\partial L}{\partial w} &= w - X^T\cdot(\alpha \odot y) = \boldsymbol{0} \nonumber \\
    \frac{\partial L}{\partial b} &= - 1^T\cdot(\alpha \odot y) = - y^T \cdot \alpha = 0  \nonumber
\end{align}
推导出如下关系:
\begin{align}
    w &= X^T\cdot(\alpha \odot y) \label{eq6} \\
    y^T \cdot \alpha &= 0 \label{eq7}
\end{align}
将\eqref{eq6}式 和 \eqref{eq7}式 代入\eqref{eq5}式中,
\begin{align}
    \mathop{\arg\max_{\alpha}} L(w, b, {\alpha}) &= \frac{1}{2}(\alpha \odot y)^T \cdot X \cdot X^T \cdot (\alpha \odot y) \nonumber \\
                      & \ \ \  - (\alpha \odot y)^T \cdot X \cdot X^T \cdot (\alpha \odot y) \nonumber \\
                      & \ \ \  - (\alpha \odot y)^T \cdot b^m  \nonumber \\
                      & \ \ \  + \boldsymbol{1}^T\cdot\alpha  \nonumber \\
                      &= -\frac{1}{2}(\alpha \odot y)^T \cdot X \cdot X^T \cdot (\alpha \odot y) + (\alpha \odot y)^T \cdot b + \boldsymbol{1}^T\cdot\alpha \nonumber \\
                      &= -\frac{1}{2}(\alpha \odot y)^T \cdot X \cdot X^T \cdot (\alpha \odot y) + \boldsymbol{1}^T\cdot\alpha \nonumber 
\end{align}
去除负号,将极大转换成极小形式:
\begin{align}
    \mathop{\arg\min_{\alpha}} L(w, b, {\alpha}) &= \frac{1}{2}(\alpha \odot y)^T \cdot X \cdot X^T \cdot (\alpha \odot y) - \boldsymbol{1}^T\cdot\alpha \label{eq8} \\
        \mathrm{ s.t. }\ \   y^T \cdot \alpha &= 0 \nonumber 
\end{align}
为了使拉格朗日对偶问题的解与原始问题解相同,需要同时满足KTT条件:
\begin{align}
    \frac{\partial L}{\partial w} &= 0 \nonumber \\
    \frac{\partial L}{\partial b} &= 0 \nonumber \\
    {\alpha} \odot  (y \odot (X \cdot w + b) - 1) &= \boldsymbol{0}^m \nonumber \\
    y \odot (X \cdot w + b) - 1 &\geq \boldsymbol{0}^m \nonumber \\
    {\alpha} &\geq \boldsymbol{0}^m  \nonumber 
\end{align}

\subsection{软间隔}
当训练数据近似线性可分,有些异常点或噪声点导致无法找到分离超平面,可以对每个数据点加一个松弛变量$\xi_i$,
从而让所有数据点均满足约束。
\begin{figure}[htbp]
    \centering
    \includegraphics[width=14cm]{svm_soft.jpg}
    \caption{SVM软间隔}\label{fig2}
\end{figure}

加入软间隔参数,每个向量点距离分类超平面距离增加$\xi_i$,同时增加一个惩罚系数C,代入\eqref{eq5}新形式如下:
\begin{align*}
    \mathop{\arg\min_{w}}  \ d &=  \frac{1}{2}w^T \cdot w + C \cdot \boldsymbol{1}^m \cdot \xi\\
    \mathrm{ s.t. }\ \   & y_i \cdot (x_i^T \cdot w + b) \geq 1	- \xi_i			\\
    & \xi_i	\geq 0		\\
\end{align*}


将其转换为拉格朗日对偶形式:
\begin{align}
    L(w, b, \xi, {\alpha}, \mu ) &= \frac{1}{2}w^T \cdot w + C \cdot {\boldsymbol{1}^T} \cdot \xi  - {\alpha} ^T \cdot  (y \odot (X \cdot w + b) - 1 + \xi) - \mu^T \cdot \xi  \label{eq9} 
\end{align}
求得对于w、b、$\xi$ 的偏导并令其为0:
\begin{align}
    \frac{\partial L}{\partial w} &= w - X^T\cdot(\alpha \odot y) = \boldsymbol{0} \nonumber \\
    \frac{\partial L}{\partial b} &= - 1^T\cdot(\alpha \odot y) = - y^T \cdot \alpha = 0  \nonumber \\
    \frac{\partial L}{\partial \xi} &= C - \alpha - u  = \boldsymbol{0} \nonumber 
\end{align}
代入 \eqref{eq9} 式中: 
\begin{align}
    \mathop{\arg\max_{\alpha}}  L(w, b, \xi, {\alpha}, \mu) &= -\frac{1}{2}(\alpha \odot y)^T \cdot X \cdot X^T \cdot (\alpha \odot y) + \boldsymbol{1}^m \cdot \alpha \nonumber \\
    \mathrm{ s.t. }\ \   &y^T \cdot \alpha = 0 \nonumber \\
    &C - \alpha - \mu \nonumber  = \boldsymbol{0} \nonumber \\
    &\alpha_i \geq 0 \nonumber \\
    &\mu_i \geq 0 \nonumber 
\end{align}
$C - \alpha - u   = \boldsymbol{0} $、$ \alpha_i \geq 0 $ 、$u_i \geq 0 $约束可以化简为:
\begin{align}
    \mathop{\arg\min_{\alpha}}  L(w, b, \xi, {\alpha}, \mu) &= \frac{1}{2}(\alpha \odot y)^T \cdot X \cdot X^T \cdot (\alpha \odot y) - \boldsymbol{1}^m \cdot \alpha \label{eq10}\\
    \mathrm{ s.t. }\ \   &y^T \cdot \alpha = 0 \nonumber \\
    & 0 \leq \alpha_i  \leq C  \iff \boldsymbol{0}^m \leq \alpha \leq \boldsymbol{C}^m \nonumber
\end{align}

为了使拉格朗日对偶问题的解与原始问题解相同,需要同时满足KTT条件:
\begin{align}
    \frac{\partial L}{\partial w} &= 0 \nonumber \\
    \frac{\partial L}{\partial b} &= 0 \nonumber \\
    \frac{\partial L}{\partial \xi} &= 0 \nonumber \\
    {\alpha} \odot  (y \odot (X \cdot w + b) - 1 + \xi) &= \boldsymbol{0}^m \nonumber \\
    y \odot (X \cdot w + b) - 1  + \xi &\geq \boldsymbol{0}^m \nonumber \\
    {\alpha} &\geq \boldsymbol{0}^m  \nonumber \\
    \mu \odot \xi &=  \boldsymbol{0}^m \nonumber \\
    \xi &\geq \boldsymbol{0}^m  \nonumber \\
    \mu &\geq \boldsymbol{0}^m  \nonumber 
\end{align}



\subsection{核函数}
近似线性可分用软间隔方式解决,然而当训练数据是非线性数据,会出现无法在原特征空间找到分离超平面。可以使用
一个非线性变换,将数据从原特征空间映射到更高维的新空间,然后在新空间中寻找线性分类超平面,这种方法被称为
核技巧。观察 \eqref{eq10} 式中计算$X \cdot X^t $,即需要计算 $x_i \cdot x_i$内积。将核技巧应用到SVM,
定义核函数为:
\begin{align*}
K(x, z) &=  \phi(x) \cdot \phi(z)
\end{align*}
即将原来的向量内积,改成先让向量映射到新空间,然后再求内积。核技巧的另外一个优点是,不需要显示的定义$\phi$
而是直接计算出 $\phi(x) \cdot \phi(z)$的结果,以高斯核函数为例:
\begin{align*}
    K(x, z) &=  \exp ( -\frac{\left\|x-z\right\|^2}{2\sigma^2 }) 
\end{align*}
则 \eqref{eq10} 式利用核技巧转化为:
\begin{align}
    \mathop{\arg\min_{\alpha}}  L(w, b, \xi, {\alpha}, \mu) &= \frac{1}{2}(\alpha \odot y)^T \cdot K(X, X) \cdot (\alpha \odot y) - \boldsymbol{1}^m \cdot \alpha \label{eq11}\\
    \mathrm{ s.t. }\ \   &y^T \cdot \alpha = 0 \nonumber \\
    & \boldsymbol{0}^m \leq \alpha \leq \boldsymbol{C}^m \nonumber \\
    &{\alpha} \odot  (y \odot (X \cdot w + b) - 1 + \xi) = \boldsymbol{0}^m \nonumber \\
    &y \odot (X \cdot w + b) - 1  + \xi \geq \boldsymbol{0}^m \nonumber \\
    &{\alpha} \geq \boldsymbol{0}^m  \nonumber \\
    &\mu \odot \xi =  \boldsymbol{0}^m \nonumber \\
    &\xi \geq \boldsymbol{0}^m  \nonumber \\
    &\mu \geq \boldsymbol{0}^m  \nonumber 
\end{align}

\subsection{序列最小最优化算法SMO}
支持向量机的拉格朗日对偶问题是一个凸二次规划问题,具有全局最优解,序列最小最优化算法即SMO算法,是高效求解
支持向量机解的一种算法。其基本思路是,如果所有变量都满足了KTT条件,那么就求得了问题的解。SMO算法过程如下:
\begin{itemize}
    \item 选择两个变量,如$\alpha_1,\alpha_2$,固定其他变量,那么原问题就变成了两个变量的二次优化问题。
    \item 由于有约束$y^T \cdot \alpha = 0$的存在,选择了两个变量,实际上自由变量只有一个。
    \item 求解两个变量的最优解,迭代知道所有变量满足KTT条件。
\end{itemize}
当选择两个变量,如$\alpha_1,\alpha_2$时,则 \eqref{eq11} 去除不包含 $\alpha_1,\alpha_2$ 的项后化简为:
\begin{align}
    \mathop{\arg\min_{\alpha_1, \alpha_2}}  W(\alpha_1, \alpha_2) &=  \label{eq12} \\
    \mathrm{ s.t. }\ \   &y^T \cdot \alpha = 0 \nonumber 
\end{align}

End!!




\end{document}                 % The input file ends with this command.

