%!TEX program = xelatex
% This is a small sample LaTeX input file (Version of 10 April 1994)
%
% Use this file as a model for making your own LaTeX input file.
% Everything to the right of a  %  is a remark to you and is ignored by LaTeX.
 
% The Local Guide tells how to run LaTeX.
 
% WARNING!  Do not type any of the following 10 characters except as directed:
%                &   $   #   %   _   {   }   ^   ~   \   
 
%\documentclass{article}        % Your input file must contain these two lines 
\documentclass[12pt, a4paper, oneside]{ctexart}
\usepackage{xeCJK}

\usepackage{indentfirst, abstract, appendix} %首行缩进
\usepackage{graphicx} %插入图片
\usepackage{amsmath, amssymb, geometry} 
\usepackage{listings, xcolor} %代码高亮
\graphicspath{{../graphics/}}
\linespread{1.2}
\geometry{left=2.5cm, right=2.5cm, top=2.5cm, bottom=2.5cm}
\title{\textbf{SVM的数学推导和Python实现}}
\author{赵新锋}
\date{\today}
\renewcommand{\abstractname}{\Large\textbf{摘要}}
\pagestyle{plain}

\lstset{ %
language=Python,                % the language of the code
basicstyle=\footnotesize,           % the size of the fonts that are used for the code
%numbers=left,                   % where to put the line-numbers
%numberstyle=\tiny\color{gray},  % the style that is used for the line-numbers
stepnumber=2,                   % the step between two line-numbers. If it's 1, each line 
                                % will be numbered
numbersep=5pt,                  % how far the line-numbers are from the code
backgroundcolor=\color{white},      % choose the background color. You must add \usepackage{color}
showspaces=false,               % show spaces adding particular underscores
showstringspaces=false,         % underline spaces within strings
showtabs=false,                 % show tabs within strings adding particular underscores
frame=single,                   % adds a frame around the code
rulecolor=\color{black},        % if not set, the frame-color may be changed on line-breaks within not-black text (e.g. commens (green here))
tabsize=2,                      % sets default tabsize to 2 spaces
captionpos=b,                   % sets the caption-position to bottom
breaklines=true,                % sets automatic line breaking
breakatwhitespace=false,        % sets if automatic breaks should only happen at whitespace
title=\lstname,                 % show the filename of files included with \lstinputlisting;
                                % also try caption instead of title
keywordstyle=\color{blue},          % keyword style
commentstyle=\it\color[RGB]{0,96,96},                % 设置代码注释的格式
stringstyle=\rmfamily\slshape\color[RGB]{128,0,0},         % string literal style
escapeinside={\%*}{*)},            % if you want to add LaTeX within your code
morekeywords={*,...}               % if you want to add more keywords to the set
}
\begin{document}               % plus the \end{document} command at the end.
\maketitle

\setcounter{page}{0}
\maketitle
\thispagestyle{empty}

\begin{abstract}
支持向量机(support vector machines, SVM)是一种分类模型,该模型在特征空间中求解间隔最大的分类超平面。当训练数据近似线性
可分时,可以通过增加软间隔学习一个线性分类器。当线性不可分时,利用核技巧,隐式的将特征空间映射到高维特征空间,从而达到线性
可分。使用序列最小最优化算法(SMO),可以快速求解模型的参数。
\par\textbf{关键词:}支持向量机; SVM; SMO; 矩阵运算; 矩阵求导;numpy;sklearn. 
\end{abstract}

\newpage
\pagenumbering{Roman}
\setcounter{page}{1}
\tableofcontents
\newpage
\setcounter{page}{1}
\pagenumbering{arabic}


\newpage
\section{数学推导与python实现}

\subsection{SVM简介}
当训练数据线性可分,可以得到一个线性超平面 $ x \cdot w + b = 0 $,将在超平面上方的归为正类,将在超平面下方的归为负类。当数据点
与超平面距离越远时,表示分类的确定性越高,这样虽然线性分类的超平面可能有无数多个,但是我们可以找到一个所有点距离超平面最大的一个
超平面。相应的决策函数为:
\begin{align*}
    f(x) = sign(x \cdot w^* + b^*)
\end{align*}
\begin{figure}[htbp]
    \centering
    \includegraphics[width=14cm]{svm1.jpg}
    \caption{SVM线性可分}\label{fig1}
\end{figure}



\subsection{最大间隔分类超平面}
$x_i^T \cdot w_0 + b_0$ 为正时,$y_i$为正,当$x_i^T \cdot w_0 + b_0$ 为负时,$y_i$为负,则可以定义
$\frac{y_i \cdot (x_i^T \cdot w_0 + b_0)}{ \left\|w_0\right\|}$为几何间隔,表示数据点距离超平面的距离。
模型最终会找到一个参数为$w_0$和$b_0$的分离超平面,所有点距离超平面的距离都大于等于d,将距离正好等于d的数据
点称之为支持向量。
\begin{align}
    \mathop{\arg\max_{w_0, b_0}} \ d &= \frac{y_0 \cdot (x_0^T \cdot w_0 + b_0)}{ \left\|w_0\right\|} 			\label{eq1}\\
	\mathrm{ s.t. }\ \   &\frac{y_i \cdot (x_i^T \cdot w_0 + b_0)}{ \left\|w_0\right\|} \geq d				\nonumber
\end{align}
将 $w_0$ 和 $b_0$ 进行一定比例的缩放
\begin{align}
	w &= \frac{w_0 }{ y_0 \cdot (x_0^T \cdot w_0 + b_0)} 			\nonumber\\
    b &= \frac{b_0 }{ y_0 \cdot (x_0^T \cdot w_0 + b_0)} 			\nonumber
\end{align}
可以将 \eqref{eq1} 式化简为:
\begin{align}
	\mathop{\arg\max_{w}} \ d &= \frac{1}{ \left\|w\right\|} 			\nonumber\\
    \Leftrightarrow   \mathop{\arg\min_{w}}  \ d &=  \left\|w\right\| \nonumber\\
    \Leftrightarrow   \mathop{\arg\min_{w}}  \ d &=  \frac{1}{2}w^T \cdot w \label{eq2}\\
    \mathrm{ s.t. }\ \   & y_i \cdot (x_i^T \cdot w + b) \geq 1				\label{eq3}
\end{align}
将\eqref{eq2} 和 \eqref{eq3} 利用拉格朗日乘数法,获得拉格朗日原始问题形式:
\begin{align}
    \mathop{\arg\min_{w,b}\max_{\alpha}} \ 	L(w, b, {\alpha}) &= \frac{1}{2}	w^T \cdot w  - {\alpha} ^T \cdot  (y \odot (X \cdot w + b) - 1) \nonumber\\
        &= \frac{1}{2}	w^T \cdot w  - (\alpha \odot y) ^T \cdot  (X \cdot w + b) + \alpha^T \cdot \boldsymbol{1}^m \nonumber\\
        &= \frac{1}{2}	w^T \cdot w  - (\alpha \odot y) ^T \cdot  (X \cdot w + b) + \boldsymbol{1}^T\cdot\alpha \label{eq4}
\end{align}
当满足KTT条件时, 拉格朗日对偶问题的解等价于\eqref{eq4} 的解:
\begin{align}
    \mathop{\arg\max_{\alpha}\min_{w,b}} \ 	L(w, b, {\alpha}) &= \frac{1}{2}	w^T \cdot w  - (\alpha \odot y) ^T \cdot  (X \cdot w + b) + \boldsymbol{1}^T\cdot\alpha \label{eq5}
\end{align}


\subsection{求解拉格朗日对偶问题}
首先求L以 w、b为参数的极小值,通过求微分得到偏导数形式。
\begin{align}
    \mathrm{d}L &= \frac{1}{2}tr[(\mathrm{d}w)^T \cdot w + w^T \cdot \mathrm{d}w) - (\alpha \odot y) ^T \cdot(X\mathrm{d}w) \nonumber \\
                & \ \ \  - (1^T\cdot(\alpha \odot y)) ^T \cdot \mathrm{d}b \nonumber\\
                & \ \ \  - (\mathrm{d}\alpha)^T \cdot (y \odot (X \cdot w + b) - 1)] \nonumber \\
                &= tr[w^T\mathrm{d}w - (X^T\cdot(\alpha \odot y))^T\mathrm{d}w - (\alpha \odot y) ^T \cdot \mathrm{d}b - (y \odot (X \cdot w + b) - 1)^T\mathrm{d}\alpha] \nonumber
\end{align}
从而得到w、b 偏导数,并令偏导数为0。
\begin{align}
    \frac{\partial L}{\partial w} &= w - X^T\cdot(\alpha \odot y) = \boldsymbol{0} \nonumber \\
    \frac{\partial L}{\partial b} &= - 1^T\cdot(\alpha \odot y) = - y^T \cdot \alpha = 0  \nonumber
\end{align}
推导出如下关系:
\begin{align}
    w &= X^T\cdot(\alpha \odot y) \label{eq6} \\
    y^T \cdot \alpha &= 0 \label{eq7}
\end{align}
将\eqref{eq6}式 和 \eqref{eq7}式 代入\eqref{eq5}式中,
\begin{align}
    \mathop{\arg\max_{\alpha}} L(w, b, {\alpha}) &= \frac{1}{2}(\alpha \odot y)^T \cdot X \cdot X^T \cdot (\alpha \odot y) \nonumber \\
                      & \ \ \  - (\alpha \odot y)^T \cdot X \cdot X^T \cdot (\alpha \odot y) \nonumber \\
                      & \ \ \  - (\alpha \odot y)^T \cdot b^m  \nonumber \\
                      & \ \ \  + \boldsymbol{1}^T\cdot\alpha  \nonumber \\
                      &= -\frac{1}{2}(\alpha \odot y)^T \cdot X \cdot X^T \cdot (\alpha \odot y) + (\alpha \odot y)^T \cdot b + \boldsymbol{1}^T\cdot\alpha \nonumber \\
                      &= -\frac{1}{2}(\alpha \odot y)^T \cdot X \cdot X^T \cdot (\alpha \odot y) + \boldsymbol{1}^T\cdot\alpha \nonumber 
\end{align}
去除负号,将极大转换成极小形式:
\begin{align}
    \mathop{\arg\min_{\alpha}} L(w, b, {\alpha}) &= \frac{1}{2}(\alpha \odot y)^T \cdot X \cdot X^T \cdot (\alpha \odot y) - \boldsymbol{1}^T\cdot\alpha \label{eq8} \\
        \mathrm{ s.t. }\ \   y^T \cdot \alpha &= 0 \nonumber 
\end{align}
为了使拉格朗日对偶问题的解与原始问题解相同,需要同时满足KTT条件:
\begin{align}
    \frac{\partial L}{\partial w} &= 0 \nonumber \\
    \frac{\partial L}{\partial b} &= 0 \nonumber \\
    {\alpha} \odot  (y \odot (X \cdot w + b) - 1) &= \boldsymbol{0}^m \nonumber \\
    y \odot (X \cdot w + b) - 1 &\geq \boldsymbol{0}^m \nonumber \\
    {\alpha} &\geq \boldsymbol{0}^m  \nonumber 
\end{align}

\subsection{软间隔}
当训练数据近似线性可分,有些异常点或噪声点导致无法找到分离超平面,可以对每个数据点加一个松弛变量$\xi_i$,
从而让所有数据点均满足约束。
\begin{figure}[htbp]
    \centering
    \includegraphics[width=14cm]{svm_soft.jpg}
    \caption{SVM软间隔}\label{fig2}
\end{figure}

加入软间隔参数,每个向量点距离分类超平面距离增加$\xi_i$,同时增加一个惩罚系数C,代入\eqref{eq5}新形式如下:
\begin{align*}
    \mathop{\arg\min_{w}}  \ d &=  \frac{1}{2}w^T \cdot w + C \cdot \boldsymbol{1}^m \cdot \xi\\
    \mathrm{ s.t. }\ \   & y_i \cdot (x_i^T \cdot w + b) \geq 1	- \xi_i			\\
    & \xi_i	\geq 0		\\
\end{align*}


将其转换为拉格朗日对偶形式:
\begin{align}
    L(w, b, \xi, {\alpha}, \mu ) &= \frac{1}{2}w^T \cdot w + C \cdot {\boldsymbol{1}^T} \cdot \xi  - {\alpha} ^T \cdot  (y \odot (X \cdot w + b) - 1 + \xi) - \mu^T \cdot \xi  \label{eq9} 
\end{align}
求得对于w、b、$\xi$ 的偏导并令其为0:
\begin{align}
    \frac{\partial L}{\partial w} &= w - X^T\cdot(\alpha \odot y) = \boldsymbol{0} \nonumber \\
    \frac{\partial L}{\partial b} &= - 1^T\cdot(\alpha \odot y) = - y^T \cdot \alpha = 0  \nonumber \\
    \frac{\partial L}{\partial \xi} &= C - \alpha - u  = \boldsymbol{0} \nonumber 
\end{align}
代入 \eqref{eq9} 式中: 
\begin{align}
    \mathop{\arg\max_{\alpha}}  L({\alpha}) &= -\frac{1}{2}(\alpha \odot y)^T \cdot X \cdot X^T \cdot (\alpha \odot y) + \boldsymbol{1}^m \cdot \alpha \nonumber \\
    \mathrm{ s.t. }\ \   &y^T \cdot \alpha = 0 \nonumber \\
    &C - \alpha - \mu \nonumber  = \boldsymbol{0} \nonumber \\
    &\alpha_i \geq 0 \nonumber \\
    &\mu_i \geq 0 \nonumber 
\end{align}
$C - \alpha - u   = \boldsymbol{0} $、$ \alpha_i \geq 0 $ 、$u_i \geq 0 $约束可以化简为:
\begin{align}
    \mathop{\arg\min_{\alpha}}  L({\alpha}) &= \frac{1}{2}(\alpha \odot y)^T \cdot X \cdot X^T \cdot (\alpha \odot y) - \boldsymbol{1}^m \cdot \alpha \label{eq10}\\
    \mathrm{ s.t. }\ \   &y^T \cdot \alpha = 0 \nonumber \\
    & 0 \leq \alpha_i  \leq C  \Leftrightarrow   \boldsymbol{0}^m \leq \alpha \leq \boldsymbol{C}^m \nonumber
\end{align}

为了使拉格朗日对偶问题的解与原始问题解相同,需要同时满足KTT条件:
\begin{align}
    \frac{\partial L}{\partial w} &= 0 \nonumber \\
    \frac{\partial L}{\partial b} &= 0 \nonumber \\
    \frac{\partial L}{\partial \xi} &= 0 \nonumber \\
    {\alpha} \odot  (y \odot (X \cdot w + b) - 1 + \xi) &= \boldsymbol{0}^m \nonumber \\
    y \odot (X \cdot w + b) - 1  + \xi &\geq \boldsymbol{0}^m \nonumber \\
    {\alpha} &\geq \boldsymbol{0}^m  \nonumber \\
    \mu \odot \xi &=  \boldsymbol{0}^m \nonumber \\
    \xi &\geq \boldsymbol{0}^m  \nonumber \\
    \mu &\geq \boldsymbol{0}^m  \nonumber 
\end{align}



\subsection{核函数}
近似线性可分用软间隔方式解决,然而当训练数据是非线性数据,会出现无法在原特征空间找到分离超平面。可以使用
一个非线性变换,将数据从原特征空间映射到更高维的新空间,然后在新空间中寻找线性分类超平面,这种方法被称为
核技巧。观察 \eqref{eq10} 式中计算$X \cdot X^t $,即需要计算 $x_i \cdot x_i$内积。将核技巧应用到SVM,
定义核函数为:
\begin{align*}
K(x, z) &=  \phi(x) \cdot \phi(z)
\end{align*}
即将原来的向量内积,改成先让向量映射到新空间,然后再求内积。核技巧的另外一个优点是,不需要显示的定义$\phi$
而是直接计算出 $\phi(x) \cdot \phi(z)$的结果,以高斯核函数为例:
\begin{align*}
    K(x, z) &=  \exp ( -\frac{\left\|x-z\right\|^2}{2\sigma^2 }) 
\end{align*}
则 \eqref{eq10} 式利用核技巧转化为:
\begin{align}
    \mathop{\arg\min_{\alpha}}  L({\alpha}) &= \frac{1}{2}(\alpha \odot y)^T \cdot K(X, X) \cdot (\alpha \odot y) - \boldsymbol{1}^m \cdot \alpha \label{eq11}\\
    \mathrm{ s.t. }\ \   &y^T \cdot \alpha = 0 \nonumber \\
    & \boldsymbol{0}^m \leq \alpha \leq \boldsymbol{C}^m \nonumber \\
    &{\alpha} \odot  (y \odot (X \cdot w + b) - 1 + \xi) = \boldsymbol{0}^m \nonumber \\
    &y \odot (X \cdot w + b) - 1  + \xi \geq \boldsymbol{0}^m \nonumber \\
    &{\alpha} \geq \boldsymbol{0}^m  \nonumber \\
    &\mu \odot \xi =  \boldsymbol{0}^m \nonumber \\
    &\xi \geq \boldsymbol{0}^m  \nonumber \\
    &\mu \geq \boldsymbol{0}^m  \nonumber 
\end{align}

\subsection{序列最小最优化算法SMO}
支持向量机的拉格朗日对偶问题是一个凸二次规划问题,具有全局最优解,序列最小最优化算法即SMO算法,是高效求解
支持向量机解的一种算法。其基本思路是,如果所有变量都满足了KTT条件,那么就求得了问题的解。SMO算法过程如下:
\begin{itemize}
    \item 选择两个变量,如$\alpha_1,\alpha_2$,固定其他变量,那么原问题就变成了两个变量的二次优化问题。
    \item 由于有约束$y^T \cdot \alpha = 0$的存在,选择了两个变量,实际上自由变量只有一个。
    \item 求解两个变量的最优解,迭代直到所有变量满足KTT条件。
    \item 迭代求解使用的解析方法,效率高
\end{itemize}
当选择两个变量,如$\alpha_1,\alpha_2$时,将其他变量看做常数:
\begin{align}
    \mathop{\arg\min_{\alpha}}  L({\alpha}) &= \frac{1}{2}(\alpha \odot y)^T \cdot K(X, X) \cdot (\alpha \odot y) - \boldsymbol{1}^m \cdot \alpha \nonumber \\
    &=  \frac{1}{2} \sum_{i = 1}^{m}\sum_{j = 1}^{m}  {\alpha}_i {\alpha}_j y_i y_j K(x_i, x_j) - \sum_{i = 1}^{m}{\alpha}_i \nonumber
\end{align}
则上式子去除不包含 $\alpha_1,\alpha_2$ 的项后化简为:
\begin{align}
    \mathop{\arg\min_{\alpha_1, \alpha_2}}  W(\alpha_1, \alpha_2) &=  \frac{1}{2}K_{11}a_1^2 + y_1y_2K_{12}a_1a_2  + \frac{1}{2}K_{22}a_2^2 \nonumber \\
        & \ \ \ + y_1a_1\sum_{i = 3}^{m}y_ia_iK_{i1}  \nonumber \\
        & \ \ \ + y_2a_2\sum_{i = 3}^{m} y_ia_iK_{i2} \nonumber \\
        & \ \ \ - (a_1 + a_2) \nonumber \\
        &=  \frac{1}{2}K_{11}a_1^2 + y_1y_2K_{12}a_1a_2  + \frac{1}{2}K_{22}a_2^2 \nonumber \\
        & \ \ \ + y_1a_1(\sum_{i = 1}^{m}y_ia_i^{old}K_{i1}  - y_1a_1^{old}K_{11} - y_2a_2^{old}K_{12} ) \nonumber \\
        & \ \ \ + y_2a_2(\sum_{i = 1}^{m} y_ia_i^{old}K_{i2} - y_1a_1^{old}K_{12} - y_2a_2^{old}K_{22} ) \nonumber \\
        & \ \ \ - (a_1 + a_2) \nonumber \\
        &=  \frac{1}{2}K_{11}a_1^2 + y_1y_2K_{12}a_1a_2  + \frac{1}{2}K_{22}a_2^2 \nonumber \\
        & \ \ \ + y_1a_1((y \odot a^{old})^T \cdot K(X, x_1) - y_1a_1^{old}K_{11} - y_2a_2^{old}K_{12} ) \nonumber \\
        & \ \ \ + y_2a_2((y \odot a^{old})^T \cdot K(X, x_2) - y_1a_1^{old}K_{12} - y_2a_2^{old}K_{22} ) \nonumber \\
        & \ \ \ - (a_1 + a_2) \nonumber 
\end{align}

利用约束$ \mathrm{ s.t. } \ \ y^T \cdot \alpha = 0 $ 得到: 
\begin{align}      
     a_1y_1 + a_2y_2 &= a_1^{old}y_1 + a_2^{old}y_2 = \zeta \nonumber\\
     a_1 &= y_1( \zeta - a_2y_2) =  a_1^{old} + a_2^{old}y_1y_2 - a_2y_1y_2 \nonumber
\end{align}
将$a_1$代入:
\begin{align}      
    \mathop{\arg\min_{\alpha_2}}  W(\alpha_2) &=  \frac{1}{2}K_{11}(y_1( \zeta - a_2y_2))^2 + y_1y_2K_{12}y_1( \zeta - a_2y_2)a_2  + \frac{1}{2}K_{22}a_2^2 \nonumber \\
        & \ \ \ + y_1y_1( \zeta - a_2y_2)((y \odot a^{old})^T \cdot K(X, x_1) - y_1a_1^{old}K_{11} - y_2a_2^{old}K_{12} ) \nonumber \\
        & \ \ \ + y_2a_2((y \odot a^{old})^T \cdot K(X, x_2) - y_1a_1^{old}K_{12} - y_2a_2^{old}K_{22} ) \nonumber \\
        & \ \ \ - (y_1( \zeta - a_2y_2) + a_2) \nonumber 
\end{align}
针对$a_2$求导,并令导数为0:
\begin{align}
    \frac{\partial W}{\partial a_2} &= K_{11}y_2( a_2y_2 - \zeta) + K_{12}(y_2\zeta - 2a_2) + K_{22}a_2 \nonumber \\
    & \ \ \ - y_2((y \odot a^{old})^T \cdot K(X, x_1) - y_1a_1^{old}K_{11} - y_2a_2^{old}K_{12} ) \nonumber \\
    & \ \ \ + y_2((y \odot a^{old})^T \cdot K(X, x_2) - y_1a_1^{old}K_{12} - y_2a_2^{old}K_{22} ) \nonumber \\
    & \ \ \ + y_1y_2 - 1 \nonumber \\
    &= a_2(K_{11} - 2K_{12} + K_{22}) \nonumber \\
    & \ \ \ - K_{11}y_2(y_1a_1^{old} + y_2a_2^{old}) + K_{12}y_2(y_1a_1^{old} + y_2a_2^{old}) \nonumber \\
    & \ \ \ + y_2((y \odot a^{old})^T \cdot K(X, x_2) - (y \odot a^{old})^T \cdot K(X, x_1) ) \nonumber \\
    & \ \ \ + y_1y_2a_1^{old}K_{11} + a_2^{old}K_{12} - y_1y_2a_1^{old}K_{12} - a_2^{old}K_{22} \nonumber \\
    & \ \ \ + y_1y_2 - 1 \nonumber \\
    &= a_2(K_{11} - 2K_{12} + K_{22}) \nonumber \\
    & \ \ \ + y_2((y \odot a^{old})^T \cdot K(X, x_2) - (y \odot a^{old})^T \cdot K(X, x_1) ) \nonumber \\
    & \ \ \ - a_2^{old}(K_{11} - 2K_{12} + K_{22}) \nonumber \\
    & \ \ \ + y_1y_2 - 1 \nonumber \\
    &= a_2(K_{11} - 2K_{12} + K_{22}) \nonumber \\
    & \ \ \ + y_2((y \odot a^{old})^T \cdot K(X, x_2) - y_2 - ((y \odot a^{old})^T \cdot K(X, x_1) - y_1)) \nonumber \\
    & \ \ \ - a_2^{old}(K_{11} - 2K_{12} + K_{22}) \nonumber \\
    &= 0 \nonumber
\end{align}
则可以推得:
\begin{align}      
    a_2^{new,unclip} &= a_2^{old} + \frac{y_2(((\alpha \odot y)^T \cdot K(X, x_1) - y_1) - ((\alpha \odot y)^T \cdot K(X, x_2)-y_2))}{(K_{11} - 2K_{12} + K_{22})} \label{eq12} 
\end{align}
考虑到$ w = X^T \cdot (a \odot y) $ 令 $ g(x) = w^T \cdot x + b = (a \odot y)^T \cdot X \cdot x + b$,将
将内积转换成核函数形式,并且定义函数$E_i$为函数$g(x)$与$y_i$的误差值:
\begin{align*}
    g(x_i) &=  (a \odot y)^T \cdot K(X, x_i) + b \\
    E_i &= g(x_i) - y_i  = (a \odot y)^T \cdot K(X, x_i) + b - y_i\\
    \eta &= K_{11} - 2K_{12} + K_{22}
\end{align*}
则 \eqref{eq12} 式得到未经剪辑的$a_2$解为:
\begin{align}      
    a_2^{new,unclip} &= a_2^{old} + \frac{y_2(E_1 - E_2)}{\eta} \label{eq13} 
\end{align}
因为有约束$a_1y_1 + a_2y_2 = \zeta = a_1^{old}y_1 + a_2^{old}y_2$ 并且 $0 \leq a_i \leq C$,
下面讨论$a_2$的上限下限$L \leq a_2 \leq H$,当$y_1 \neq y_2$时:
\begin{align*}
    a_2 - a_1 &= a_2^{old} - a_1^{old} \\
    L &= max(0, a_2^{old} - a_1^{old}) \\
    H &= min(C, C + a_2^{old} - a_1^{old})
\end{align*}
当$y_1 = y_2$时:
\begin{align*}
    a_2 + a_1 &= a_2^{old} + a_1^{old} \\
    L &= max(0, a_2^{old} + a_1^{old} - C) \\
    H &= min(C, a_2^{old} + a_1^{old})
\end{align*}
则$a_2$经剪辑后解为:
\begin{align*}
    a_2^{new} &= 
\left\{
    \begin{array}{lr}
    H, & a_2^{new,unclip} > H  \\
    a_2^{new,unclip}, & L \leq a_2^{new,unclip} \leq H\\
    L, & a_2^{new,unclip} < L  
    \end{array}
\right.
\end{align*}
对于偏置$b$,由于KTT条件约束有:
\begin{align*}
    {\alpha} \odot  (y \odot (X \cdot w + b) - 1) &= \boldsymbol{0}^m 
\end{align*}
对于任一$a_i > 0$有  $(a^{old} \odot y))^T \cdot K(X, x_i) + b = y_i $ 整理后得到,当$0 \leq a_1 \leq C$时:
\begin{align*}
    b_1^{new} &= y_1 - (a^{new} \odot y))^T \cdot K(X, x_1) \\
    &= y_1 - (a^{old} \odot y))^T \cdot K(X, x_1) + a_1^{old}y_1K_{11} + a_2^{old}y_2K_{12} - a_1^{new}y_1K_{11} - a_2^{new}y_2K_{12} 
\end{align*}
当$0 \leq a_2 \leq C$时:
\begin{align*}
    b_2^{new} &= y_2 - (a^{new} \odot y))^T \cdot K(X, x_2) \\
    &= y_2 - (a^{old} \odot y))^T \cdot K(X, x_2) + a_1^{old}y_1K_{12} + a_2^{old}y_2K_{22} - a_1^{new}y_1K_{12} - a_2^{new}y_2K_{22}
\end{align*}
若$a_1$ 、$a_2$都满足条件,那么$b_1^{new} = a_2^{new}$。\\
当所有$a_i$满足KTT条件时,模型得解。

\section{总结}
\subsection{线性可分支持向量机}
当训练数据线性可分时,支持向量机可以得到一个最大间隔的分离超平面,若近似可分,则可以添加软间隔的方式,若线性
不可分,则可以利用核技巧确保可以找到最大间隔分离超平面。由于得到的分离间隔最大,支持向量机有较好的鲁棒性。
\subsection{模型训练}
通过拉格朗日对偶变换,通过KTT条件,可以将原始问题转换为拉格朗日对偶问题,从而自然的引入核函数。SMO是常用的模型训练算法,
其本质上是凸二次规划问题,当所有变量都满足KTT条件,可以得到最佳解。SMO的计算过程直接利用解析解,求解速度较快。


\newpage
\begin{thebibliography}{99}
    \bibitem{b}Ian Goodfellow / Yoshua Bengio / Aaron Courville . \emph{Deep Learning}[M]. 北京:清华大学出版社,2017-08-01.
    \bibitem{b}张贤达. \emph{矩阵分析与应用}[M]. 北京:人民邮电出版社,2013-11-01.
    \bibitem{c}李航. \emph{统计学习方法}[M]. 北京:清华大学出版社,2012-03.
    \bibitem{d}David C. Lay / Steven R. Lay / Judi J. McDonald . \emph{线性代数及其应用}[M]. 北京:机械工业出版社,2018-07.
    \bibitem{e}史蒂文·J. 米勒 . \emph{普林斯顿概率论读本}[M]. 北京:人民邮电出版社,2020-08.
\end{thebibliography}


\newpage
\begin{appendices}
\section{支持向量机的python源码}
支持向量机的python源码实现,只使用numpy库:

\begin{lstlisting}
# -*- coding: utf8 -*-
import math
import time
import numpy as np 
import sklearn
from sklearn import datasets
from sklearn.datasets import load_breast_cancer
from sklearn import preprocessing
import matplotlib.pyplot as plt
from sklearn.metrics import r2_score
dataset = sklearn.datasets.load_breast_cancer()
dataset.target[dataset.target == 0] = -1 #0/1 标记替换成1/-1标记

class GSKernal:
    def __init__(self):
        self.sigma = 10
    def cal(self, x, z):
        result = None
        if len(x.shape) == 2:
            result = np.linalg.norm(x-z, axis=1) ** 2
        else:
            result = np.linalg.norm(x-z) ** 2
        return np.exp(-1 * result / (2 * self.sigma**2))

class SVM:
    def __init__(self, kn = None, C = 1, toler = 0.001):
        self.w = None #w
        self.b = None #b
        if not kn:
            kn = GSKernal()
        self.kn= kn
        self.C = C
        self.toler = toler
        self.alpha = None
        self.X = None
        self.Y = None
        self.kCache = {}
        return
    def calK(self, i, j):
        key = ''
        if i <= j:
            key = '%d_%d'%(i, j)
        else:
            key = '%d_%d'%(j, i)
        ret = self.kCache.get(key)
        if ret:
            return ret
        x1 = self.X[i]
        x2 = self.X[j]
        ret = self.kn.cal(x1, x2)
        self.kCache[key] = ret
        return ret
    def calK2(self, j):
        key = 'All_%d'%(j)
        ret = self.kCache.get(key)
        if ret:
            return ret['data']
        ret = np.zeros(self.X.shape[0])
        x2 = self.X[j]
        for i in range(self.X.shape[0]):
            x1 = self.X[i]
            ret[i] = self.kn.cal(x1, x2)
        self.kCache[key] = {'data':ret}
        return ret
        
    def isZero(self, v):
        return math.fabs(v) < self.toler
    def calGxi(self, i):
        xi = self.X[i]
        ay = np.multiply(self.alpha, self.Y)
        gxi = np.dot(ay, self.calK2(i)) + self.b
        return gxi
    def calcEi(self, i):
        gxi = self.calGxi(i)
        return gxi - self.Y[i]
    def isSatisfyKKT(self, i):
        gxi = self.calGxi(i)
        yi = self.Y[i]
        if self.isZero(self.alpha[i]) and (yi * gxi >= 1):
            return True
        elif self.isZero(self.alpha[i] - self.C) and (yi * gxi <= 1):
            return True
        elif (self.alpha[i] > -self.toler) and (self.alpha[i] < (self.C + self.toler)) \
                and self.isZero(yi * gxi - 1):
            return True

        return False
    def getAlphaJ(self, E1, i):
        E2 = 0
        maxE1_E2 = -1
        maxIndex = -1

        for j in range(self.X.shape[0]):
            if j == i:
                continue
            E2_tmp = self.calcEi(j)
            if E2_tmp == 0:
                continue
            if math.fabs(E1 - E2_tmp) > maxE1_E2:
                maxE1_E2 = math.fabs(E1 - E2_tmp)
                E2 = E2_tmp
                maxIndex = j
        if maxIndex == -1:
            maxIndex = i
            while maxIndex == i:
                maxIndex = int(random.uniform(0, self.X.shape[0]))
            E2 = self.calcEi(maxIndex)

        return E2, maxIndex

    def fit(self, X, Y, iterMax = 10):
        self.kCache.clear()
        scalerX = sklearn.preprocessing.StandardScaler().fit(X)#StandardScaler
        #Y = Y.reshape(-1, 1)
        X = scalerX.transform(X)
        
        w = np.zeros(X.shape[1])
        self.b = 0
        a = np.zeros(X.shape[0])
        self.alpha = a
        self.X = X
        self.Y = Y

        iterStep = 0; parameterChanged = 1

        calTims = 0
        while (iterStep < iterMax) and (parameterChanged > 0):
            iterStep += 1
            parameterChanged = 0

            for i in range(self.X.shape[0]):
                if self.isSatisfyKKT(i):
                    continue
                E1 = self.calcEi(i)
                E2, j = self.getAlphaJ(E1, i)
                                
                y1 = self.Y[i]
                y2 = self.Y[j]
                x1 = self.X[i]
                x2 = self.X[j]
                a1_old = a[i]
                a2_old = a[j]
                
                if y1 != y2:
                    L = max(0, a2_old - a1_old)
                    H = min(self.C, self.C + a2_old - a1_old)
                else:
                    L = max(0, a2_old + a1_old - self.C)
                    H = min(self.C, a2_old + a1_old)
                if L == H:
                    continue
                
                k11 = self.calK(i, i)
                k12 = self.calK(i, j)
                k21 = k12
                k22 = self.calK(j, j)
                      
                ay = np.multiply(a, self.Y)
                ayk1 = np.dot(ay, self.calK2(i))
                ayk2 = np.dot(ay, self.calK2(j))
                a2_new = a2_old + (y2 * (ayk1 - y1 - (ayk2 - y2))) / (k11 - 2*k12 + k22)
                a1_new = a1_old + y1 * y2 * (a2_old - a2_new)
                
                b1New = -1 * E1 - y1 * k11 * (a1_new - a1_old) \
                            - y2 * k21 * (a2_new - a2_old) + self.b
                b2New = -1 * E2 - y1 * k12 * (a1_new - a1_old) \
                        - y2 * k22 * (a2_new - a2_old) + self.b
                bNew = 0
                if (a1_new > 0) and (a1_new < self.C):
                    bNew = b1New
                elif (a2_new > 0) and (a2_new < self.C):
                    bNew = b2New
                else:
                    bNew = (b1New + b2New) / 2

                self.alpha[i] = a1_new
                self.alpha[j] = a2_new
                self.b = bNew
                if math.fabs(a2_new - a2_old) >= 0.00001:
                    parameterChanged += 1
                calTims += 1
                
                if calTims % 50 == 0:
                    print('train iter:%d SMO times:%d'%(iterStep, calTims))
                #time.sleep(1)
                #return
        Yo = self.predict(self.X)
        #print(Yo)
        score = r2_score(Yo, self.Y)
        print('score', score)
        return
    def predict(self, X):
        if len(X.shape) == 1:
            return self.predictOne(X)
        ret = np.zeros(X.shape[0])
        for i in range(X.shape[0]):
            ret[i] = self.predictOne(X[i])
        return ret
    def predictOne(self, x):
        ret = np.zeros(self.X.shape[0])
        x2 = x
        for i in range(self.X.shape[0]):
            x1 = self.X[i]
            ret[i] = self.kn.cal(x1, x2)
        ay = np.multiply(self.alpha, self.Y)
        gxi = np.dot(ay, ret) + self.b
        if gxi > 0:
            return 1
        return -1

if __name__ == '__main__':
    svm = SVM()
    print('data', dataset.data.shape)
    svm.fit(dataset.data, dataset.target)

\end{lstlisting}

\end{appendices}

\end{document}                 % The input file ends with this command.

